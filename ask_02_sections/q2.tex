\subsubsection{Ασταθής πολυδονητής Α}
	Το τερματικό 2 του 555 είναι το trigger $\mathrm{TR}$ και το τερματικό 6 είναι το threshold $\mathrm{TH}$. Όταν το 555 λαμβάνει ενεργό σήμα $\overline{\mathrm{TR}}$ η έξοδος του περνάει στο HIGH, κοντά στην τάση τροφοδοσίας $V_{CC}$ και παραμένει εκεί για χρόνο $t_{on}$ έως ότου να παρουσιαστεί ενεργό σήμα $\mathrm{TH}$. Τότε, η έξοδος του 555 περνάει στο LOW, κοντά στη γείωση.\cite{artofelectronics}\par
	Το $\overline{\mathrm{TR}}$ ενεργοποιείται από τάση μικρότερη του $\frac{1}{3}V_{CC}$, ενώ το $\mathrm{TH}$ από τάση μεγαλύτερη των $\frac{2}{3}V_{CC}$.\cite{artofelectronics}\cite{sedra}\cite{scherz}\par
	Ο πυκνωτής $C_1$ αρχίζει να φορτίζεται προς $V_{CC}$ μόλις το κύκλωμα συνδεθεί στην τροφοδοσία.\cite{scherz} Η φόρτισή του, λόγω των δύο διόδων, γίνεται μόνο μέσω του ωμικού αντιστάτη $R_1$. Η έξοδος του χρονιστή 555 βρίσκεται σε στάθμη HIGH όσο ο πυκνωτής φορτίζεται. Μόλις ο πυκνωτής $C_1$ ξεπεράσει τα $\frac{2}{3}V_{CC}$ το σήμα $\mathrm{TH}$ γίνεται ενεργό και το $\overline{\mathrm{TR}}$ απενεργοποιείται, οδηγώντας την έξοδο σε στάθμη LOW για χρονικό διάστημα $t_{off}$, και ο πυκνωτής αρχίζει να εκφορτίζεται, μέσω του $R_2$, προς τη γείωση.\cite{artofelectronics}\par
	Βάσει των παραπάνω, η τάση του πυκνωτή $C_1$ είναι $\frac{1}{3}V_{CC}\leqslant V_{C1}\leqslant\frac{2}{3}V_{CC}$ και η περίοδος του παλμού\footnote{Εάν δεν υπήρχαν οι δίοδοι θα ήταν $T=0.693\(R_1+2R_2\)$.\cite{artofelectronics}\cite{sedra}\cite{scherz}} είναι $T=t_s=0.693(R_1+R_2)\cdot C_1$. Εφόσον είναι $T=t_s=t_{on}+t_{off}$ και οι δίοδοι ορίζουν δύο ξεχωριστές διαδρομές για τη φόρτιση και την εκφόρτιση του πυκνωτή, θα είναι $t_{on}=0.693R_1\cdot C_1$ και $t_{off}=0.693R_2\cdot C_1$.\par
	% TODO:
	% + σχόλιο για το duty cycle
	% + διάγραμμα V1

\subsubsection{Ασταθής πολυδονητής Β}