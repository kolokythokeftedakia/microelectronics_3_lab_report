\subsubsection{Ασταθής πολυδονητής Α}
	Το τερματικό 2 του 555 είναι το trigger $\mathrm{TR}$ και το τερματικό 6 είναι το threshold $\mathrm{TH}$. Όταν το 555 λαμβάνει ενεργό σήμα $\overline{\mathrm{TR}}$ η έξοδος του περνάει στο HIGH, κοντά στην τάση τροφοδοσίας $V_{CC}$ και παραμένει εκεί για χρόνο $t_{on}$ έως ότου να παρουσιαστεί ενεργό σήμα $\mathrm{TH}$. Τότε, η έξοδος του 555 περνάει στο LOW, κοντά στη γείωση.\cite{artofelectronics}\par
	Το $\overline{\mathrm{TR}}$ ενεργοποιείται από τάση μικρότερη του $\frac{1}{3}V_{CC}$, ενώ το $\mathrm{TH}$ από τάση μεγαλύτερη των $\frac{2}{3}V_{CC}$.\cite{artofelectronics}\cite{sedra}\cite{scherz}\par
	Ο πυκνωτής $C_1$ αρχίζει να φορτίζεται προς $V_{CC}$ μόλις το κύκλωμα συνδεθεί στην τροφοδοσία.\cite{scherz} Η φόρτισή του, λόγω των δύο διόδων, γίνεται μόνο μέσω του ωμικού αντιστάτη $R_1$. Η έξοδος του χρονιστή 555 βρίσκεται σε στάθμη HIGH όσο ο πυκνωτής φορτίζεται. Μόλις ο πυκνωτής $C_1$ ξεπεράσει τα $\frac{2}{3}V_{CC}$ το σήμα $\mathrm{TH}$ γίνεται ενεργό και το $\overline{\mathrm{TR}}$ απενεργοποιείται, οδηγώντας την έξοδο σε στάθμη LOW για χρονικό διάστημα $t_{off}$, και ο πυκνωτής αρχίζει να εκφορτίζεται, μέσω του $R_2$, προς τη γείωση.\cite{artofelectronics}\par
	Βάσει των παραπάνω, η τάση του πυκνωτή $C_1$ είναι $\frac{1}{3}V_{CC}\leqslant V_{C1}\leqslant\frac{2}{3}V_{CC}$ και η περίοδος του παλμού\footnote{Εάν δεν υπήρχαν οι δίοδοι θα ήταν $T=0.693\cdot\(R_1+2R_2\)$.\cite{artofelectronics}\cite{sedra}\cite{scherz}} είναι $T=t_s=0.693\cdot(R_1+R_2)\cdot C_1$. Εφόσον είναι $T=t_s=t_{on}+t_{off}$ και οι δίοδοι ορίζουν δύο ξεχωριστές διαδρομές για τη φόρτιση και την εκφόρτιση του πυκνωτή, θα είναι $t_{on}=0.693\cdot R_1\cdot C_1$ και $t_{off}=0.693\cdot R_2\cdot C_1$. Τέλος, ο κύκλος εργασίας (duty cycle), $\sfrac{t_{on}}{t_s}$, είναι προφανές πως ισούται με $\sfrac{R_1}{\(R_1+R_2\)}$.\par
	% TODO:
	% + διάγραμμα V1
\subsubsection{BJT transistor στο reset του timer 555}

\subsubsection{Ασταθής πολυδονητής Β}
	Ο ασύμμετρος ασταθής πολυδονητής Β υλοποιείται χρήσει τελεστικού ενισχυτή μA741. Έστω $L_+$ η υψηλή στάθμη της εξόδου του πολυδονητή και $L_-$ η χαμηλή στάθμη του πολυδονητή. Οι δύο αυτές καταστάσεις της εξόδου χαρακτηρίζονται ως ημισταθείς \textsl{(quasi-stable states)}\cite{sedra}και η διάρκειά τους εξαρτάται από τις χρονικές σταθερές του δικτύου $R_AR_BC_2$ και από τις τάσεις κατωφλίου του πολυδονητή.\cite{sedra}\par
	Έστω πως η λειτουργία του κυκλώματος ξεκινά με την έξοδο του πολυδονητή στην υψηλή στάθμη $L_+$, είναι δηλαδή $V_2=L_+$. Τότε, μέσω της $R_A$ ο πυκνωτής $C_2$ αρχίζει να φορτίζεται προς $L_+$. Επομένως, η τάση στην αναστρέφουσα είσοδο του τελεστικού ενισχυτή αυξάνεται εκθετικά προς $L_+$ ως εξής
	\begin{equation}
		\label{eq:ask2:B:inverting:high}
		v_-=L_+-\(L_+-\beta L_-\)\exp{\(-\sfrac{t}{\tau_+}\)},
	\end{equation}
	όπου $\beta=\frac{R_i}{R_f+R_i}$\cite{sedra}\cite{jaeger} που προκύπτει από τον διαιρέτη τάσης μεταξύ της εξόδου και της μη αναστρέφουσας εισόδου του τελεστικού ενισχυτή και $\tau_+=R_A\cdot C_2$ η χρονική σταθερά φόρτισης του πυκνωτή. Ταυτόχρονα, η τάση στην μη αναστρέφουσα είσοδο του τελεστικού ενισχυτή είναι $v_+=\beta\cdot L_+$.\par
	Μόλις η τάση στα άκρα του πυκνωτή φτάσει την άνω τάση κατωφλίου $V_{HTP}=\beta\cdot L_+$\cite{sedra} ο πολυδονητής αλλάζει κατάσταση και η έξοδός του περνά στη χαμηλή στάθμη $V_2=L_-$. Η τάση στη μη αναστρέφουσα είσοδο του τελεστικού είναι $v_+=\beta\cdot L_+$ και ο πυκνωτής $C_2$ εκφορτίζεται, μέσω της $R_B$ προς $L_-$. Επομένως, η τάση στην αναστρέφουσα είσοδο του τελεστικού ενισχυτή μειώνεται εκθετικά προς $L_-$ ως εξής
	\begin{equation}
		\label{eq:ask2:B:inverting:low}
		v_-=L_--\(L_--\beta L_+\)\exp{\(-\sfrac{t}{\tau_-}\)},
	\end{equation}
	όπου $\tau_-=R_B\cdot C_2$ η χρονική σταθερά εκφόρτισης του πυκνωτή. Μόλις η τάση στην αναστρέφουσα είσοδο φτάσει την κάτω τάση κατωφλίου $V_{LTP}=\beta\cdot L_-$\cite{sedra} ο πολυδονητής αλλάζει και πάλι κατάσταση και η έξοδός του, $V_2$, περνάει στην υψηλή στάθμη $L_+$.\par
	Από τη σχέση \eqref{eq:ask2:B:inverting:high} προκύπτει πως η έξοδος του πολυδονητή παραμένει στην θετική στάθμη για χρόνο
	\begin{equation}
		t_H=R_A\cdot C_2\cdot\ln{\left[\frac{1-\beta\(\sfrac{L_-}{L_+}\)}{1-\beta}\right]}.
	\end{equation}

	Ομοίως, από τη σχέση \eqref{eq:ask2:B:inverting:low} προκύπτει πως η έξοδος του πολυδονητή παραμένει στην αρνητική στάθμη για χρόνο
	\begin{equation}
		t_L=R_B\cdot C_2\cdot\ln{\left[\frac{1-\beta\(\sfrac{L_+}{L_-}\)}{1-\beta}\right]}.
	\end{equation}

	Η δημιουργία δύο διαφορετικών μονοπατιών φόρτισης και εκφόρτισης του πυκνωτή $C_2$, χρήσει των διόδων, επιτρέπει την προσαρμογή του duty cycle σε οποιοδήποτε επιθυμητό ποσοστό.\par

\subsubsection{Ολοκληρωτής}

\subsubsection{Συνολική λειτουργία}