\subsubsection{Ασταθής πολυδονητής Α}
	Το τερματικό 2 του 555 είναι το trigger $\mathrm{TR}$ και το τερματικό 6 είναι το threshold $\mathrm{TH}$. Όταν το 555 λαμβάνει ενεργό σήμα $\overline{\mathrm{TR}}$ η έξοδος του περνάει στο HIGH, κοντά στην τάση τροφοδοσίας $V_{CC}$ και παραμένει εκεί για χρόνο $t_{\mathrm{on}}$ έως ότου να παρουσιαστεί ενεργό σήμα $\mathrm{TH}$. Τότε, η έξοδος του 555 περνάει στο LOW, κοντά στη γείωση.\cite{artofelectronics}\par
	Το $\overline{\mathrm{TR}}$ ενεργοποιείται από τάση μικρότερη του $\frac{1}{3}V_{CC}$, ενώ το $\mathrm{TH}$ από τάση μεγαλύτερη των $\frac{2}{3}V_{CC}$.\cite{artofelectronics}\cite{sedra}\cite{scherz}\par
	Ο πυκνωτής $C_1$ αρχίζει να φορτίζεται προς $V_{CC}$ μόλις το κύκλωμα συνδεθεί στην τροφοδοσία.\cite{scherz} Η φόρτισή του, λόγω των δύο διόδων, γίνεται μόνο μέσω του ωμικού αντιστάτη $R_1$. Η έξοδος του χρονιστή 555 βρίσκεται σε στάθμη HIGH όσο ο πυκνωτής φορτίζεται. Μόλις ο πυκνωτής $C_1$ ξεπεράσει τα $\frac{2}{3}V_{CC}$ το σήμα $\mathrm{TH}$ γίνεται ενεργό και το $\overline{\mathrm{TR}}$ απενεργοποιείται, οδηγώντας την έξοδο σε στάθμη LOW για χρονικό διάστημα $t_{\mathrm{off}}$, και ο πυκνωτής αρχίζει να εκφορτίζεται, μέσω του $R_2$, προς τη γείωση.\cite{artofelectronics}\par
	Βάσει των παραπάνω, η τάση του πυκνωτή $C_1$ είναι $\frac{1}{3}V_{CC}\leqslant v_{C_1}\leqslant\frac{2}{3}V_{CC}$ και η περίοδος του παλμού\footnote{Εάν δεν υπήρχαν οι δίοδοι θα ήταν $T=0.693\cdot\(R_1+2R_2\)$.\cite{artofelectronics}\cite{sedra}\cite{scherz}} είναι $T=t_s=0.693\cdot(R_1+R_2)\cdot C_1$. Εφόσον είναι $T=t_s=t_{\mathrm{on}}+t_{\mathrm{off}}$ και οι δίοδοι ορίζουν δύο ξεχωριστές διαδρομές για τη φόρτιση και την εκφόρτιση του πυκνωτή, θα είναι $t_{\mathrm{on}}=0.693\cdot R_1\cdot C_1$ και $t_{\mathrm{off}}=0.693\cdot R_2\cdot C_1$. Τέλος, ο κύκλος εργασίας (duty cycle), $\sfrac{t_{\mathrm{on}}}{t_s}$, είναι προφανές πως ισούται με $\sfrac{R_1}{\(R_1+R_2\)}$.\par
	% TODO:
	% + διάγραμμα V1
\subsubsection{BJT $T_2$}
	Το BJT $T_2$ εξασφαλίζει τον συγχρονισμό μεταξύ των δύο ταλαντωτών Α και Β. Ο χρονιστής 555 λειτουργεί όσο το reset (R) (ακροδέκτης 4) είναι συνδεδεμένο στην τροφοδοσία, $V_{CC}$. Προκειμένου να σταματήσουμε την παραγωγή των παλμών του ασταθούς Α και να ξεκινήσει εκ νέου η παραγωγή της κλίμακας, $v_{\mathrm{out}}$, πρέπει το reset να πάρει τιμή κοντά στη γείωση. Δηλαδή πρέπει να είναι ενεργό το $\overline{\mathrm{R}}$.\par
	Όταν η έξοδος του ασταθούς Β βρίσκεται στην υψηλή στάθμη, κατά συνέπεια και η $v_3$, σταματά η παραγωγή της κλίμακας $u_{\mathrm{out}}$. Επομένως, μέσω αντίστασης, η $v_3$ συνδέεται στην βάση του $T_2$ το οποίο ξεκινά να άγει στην περιοχή του κορεσμού και εφαρμόζει στον ακροδέκτη reset του 555 μία τάση κοντά στη γείωση, $v_{CE}$.

\subsubsection{Ασταθής πολυδονητής Β}
	Ο ασύμμετρος ασταθής πολυδονητής Β υλοποιείται χρήσει τελεστικού ενισχυτή μA741. Έστω $L_+$ η υψηλή στάθμη της εξόδου του πολυδονητή και $L_-$ η χαμηλή στάθμη του πολυδονητή. Οι δύο αυτές καταστάσεις της εξόδου χαρακτηρίζονται ως ημισταθείς \textsl{(quasi-stable states)}\cite{sedra}και η διάρκειά τους εξαρτάται από τις χρονικές σταθερές του δικτύου $R_AR_BC_2$ και από τις τάσεις κατωφλίου του πολυδονητή.\cite{sedra}\par
	Έστω πως η λειτουργία του κυκλώματος ξεκινά με την έξοδο του πολυδονητή στην υψηλή στάθμη $L_+$, είναι δηλαδή $v_2=L_+$. Τότε, μέσω της $R_A$ ο πυκνωτής $C_2$ αρχίζει να φορτίζεται προς $L_+$. Επομένως, η τάση στην αναστρέφουσα είσοδο του τελεστικού ενισχυτή αυξάνεται εκθετικά προς $L_+$ ως εξής
	\begin{equation}
		\label{eq:ask2:B:inverting:high}
		v_-=L_+-\(L_+-\beta L_-\)\exp{\(-\sfrac{t}{\tau_+}\)},
	\end{equation}
	όπου $\beta=\frac{R_i}{R_f+R_i}$\cite{sedra}\cite{jaeger} που προκύπτει από τον διαιρέτη τάσης μεταξύ της εξόδου και της μη αναστρέφουσας εισόδου του τελεστικού ενισχυτή και $\tau_+=R_A\cdot C_2$ η χρονική σταθερά φόρτισης του πυκνωτή. Ταυτόχρονα, η τάση στην μη αναστρέφουσα είσοδο του τελεστικού ενισχυτή είναι $v_+=\beta\cdot L_+$.\par
	Μόλις η τάση στα άκρα του πυκνωτή φτάσει την άνω τάση κατωφλίου $V_{UTP}=\beta\cdot L_+$\cite{sedra} ο πολυδονητής αλλάζει κατάσταση και η έξοδός του περνά στη χαμηλή στάθμη $v_2=L_-$. Η τάση στη μη αναστρέφουσα είσοδο του τελεστικού είναι $v_+=\beta\cdot L_+$ και ο πυκνωτής $C_2$ εκφορτίζεται, μέσω της $R_B$ προς $L_-$. Επομένως, η τάση στην αναστρέφουσα είσοδο του τελεστικού ενισχυτή μειώνεται εκθετικά προς $L_-$ ως εξής
	\begin{equation}
		\label{eq:ask2:B:inverting:low}
		v_-=L_--\(L_--\beta L_+\)\exp{\(-\sfrac{t}{\tau_-}\)},
	\end{equation}
	όπου $\tau_-=R_B\cdot C_2$ η χρονική σταθερά εκφόρτισης του πυκνωτή. Μόλις η τάση στην αναστρέφουσα είσοδο φτάσει την κάτω τάση κατωφλίου $V_{LTP}=\beta\cdot L_-$\cite{sedra} ο πολυδονητής αλλάζει και πάλι κατάσταση και η έξοδός του, $v_2$, περνάει στην υψηλή στάθμη $L_+$.\par
	Από τη σχέση \eqref{eq:ask2:B:inverting:high} αντικαθιστώντας $v_-=\beta\cdot L_+=V_{UTP}$ προκύπτει πως η έξοδος του πολυδονητή παραμένει στην θετική στάθμη για χρόνο
	\begin{equation}
		t_{\mathrm{H}}=R_A\cdot C_2\cdot\ln{\left[\frac{1-\beta\(\sfrac{L_-}{L_+}\)}{1-\beta}\right]}.
	\end{equation}

	Ομοίως, από τη σχέση \eqref{eq:ask2:B:inverting:low} προκύπτει πως η έξοδος του πολυδονητή παραμένει στην αρνητική στάθμη για χρόνο
	\begin{equation}
		t_{\mathrm{L}}=R_B\cdot C_2\cdot\ln{\left[\frac{1-\beta\(\sfrac{L_+}{L_-}\)}{1-\beta}\right]}.
	\end{equation}

	Στην περίπτωση που $L+\cong -L_-$ τότε η περίοδος $T$ του πολυδονητή είναι
	\begin{equation*}
		T=t_{\mathrm{H}}+t_{\mathrm{L}}=\(R_A+R_B\)\cdot C_2\cdot\ln{\(\frac{1+\beta}{1-\beta}\)}.
	\end{equation*}

	Η δημιουργία δύο διαφορετικών μονοπατιών φόρτισης και εκφόρτισης του πυκνωτή $C_2$, χρήσει των διόδων, επιτρέπει την προσαρμογή του duty cycle σε οποιοδήποτε επιθυμητό ποσοστό.\par

\subsubsection{Ολοκληρωτής}
	Για τα ρεύματα στο σχήμα του ολοκληρωτή ισχύει $i_i=i_f+i_C\Longleftrightarrow i_C=i_i-i_f$ (Kirchhoff's current law). Από τον νόμο του Ohm εύκολα προκύπτει πως
	\begin{equation*}
		i_i=\frac{v_1-v_c}{R}\quad\land\quad i_f=\frac{v_c-v_{\mathrm{out}}}{R},\quad R=10\kohm.
	\end{equation*}
	Συνεπώς,
	\begin{equation*}
		i_C=\frac{v_1-v_c-\(v_c-v_{\mathrm{out}}\)}{R}=\frac{v_1+v_{\mathrm{out}}-2v_c}{R}.
	\end{equation*}

	Με αντικατάσταση της παραπάνω έκφρασης του ρεύματος $i_C$ στη σχέση τάσης---ρεύματος πυκνωτή $v_c=\displaystyle{\frac{1}{C}\int{i_C\dd{t}}}$ έχουμε
	\begin{equation*}
		v_c=\frac{1}{C}\int{\frac{v_1+v_{\mathrm{out}}-2v_c}{R}\dd{t}}
	\end{equation*}
	ή
	\begin{equation}\label{eq:ask2:vc:int}
		v_c=\frac{1}{R\cdot C}\int{\(v_1+v_{\mathrm{out}}-2v_c\)\dd{t}}
	\end{equation}
	Aπό τον διαιρέτη τάση στο δίκτυο ανάδρασης της αναστρέφουσας εισόδου του τελεστικού ενισχυτή προκύπτει πως $v_-=\frac{R}{R+R}v_{\mathrm{out}}\Longleftrightarrow v_-=\frac{1}{2}v_{\mathrm{out}}$. Ο τελεστικός ενισχυτής θεωρούμε πως βρίσκεται στη γραμμική περιοχή λειτουργίας του. Επομένως, η τάση της μη αναστρέφουσας εισόδου του είναι ίση με την τάσης της αναστρέφουσας εισόδου, δηλαδή $V_+=V_-=v_c$. Τότε θα είναι
	\begin{equation}\label{eq:ask2:vout:vc}
		v_c=\frac{1}{2}v_{\mathrm{out}}.
	\end{equation}
	Αντικαθιστώντας τη σχέση \eqref{eq:ask2:vout:vc} στη σχέση \eqref{eq:ask2:vc:int} έχουμε
	\begin{equation}
		\label{eq:ask2:integrator:vout:v1}
		v_{\mathrm{out}}(t)=\frac{2}{R\cdot C}\int{v_1(t)\dd{t}}.
	\end{equation}
	Η σχέση \eqref{eq:ask2:integrator:vout:v1} είναι που δικαιολογεί τη λειτουργία του κυκλώματος ως ολοκληρωτή.\par

\subsubsection{BJT $T_1$}
	 Όσο η έξοδος του του ασταθούς Β είναι στη χαμηλή στάθμη, το $T_1$ είναι σε αποκοπή. Όταν η έξοδος του ασταθούς Β περνά στην υψηλή στάθμη, το $T_1$ ξεκινά να άγει και εισέρχεται στον κορεσμό και ο πυκνωτής $C$ αρχίζει την εκφόρτιση προς τη γείωση.\par

\subsubsection{Συνολική λειτουργία}
	Στην έξοδο του κυκλώματος, $v_{\mathrm{out}}$, περιοδικά εμφανίζεται κλίμακα τάσης.\par
	Η έξοδος του ασταθούς πολυδονητή Α συνδέεται μέσω αντίστασης R στη μη αναστρέφουσα είσοδο του τελεστικού ενισχυτή του ολοκληρωτή και στον πυκνωτή $C$ του ολοκληρωτή. Όσο η έξοδος του πολυδονητή Α, $v_1$, είναι στην υψηλή στάθμη ο πυκνωτής $C$, του κυκλώματος του ολοκληρωτή, φορτίζεται και εκτελείται η πράξη της ολοκλήρωσης. Όσο η έξοδος του πολυδονητή Α είναι στο μηδέν το κύκλωμα του ολοκληρωτή διατηρεί σταθερή την τάση στην έξοδό του. Επομένως, η συνολική διάρκεια ενός βήματος είναι $t_s=t_{\mathrm{on}}+t_{\mathrm{off}}$ η οποία ισούται με την περίοδο του ασταθούς Α.\par
	Η παραγωγή της κλίμακας πραγματοποιείται όταν η τάση $v_3$ είναι μηδέν. Εξαιτίας της διόδου στην έξοδο του ασταθούς πολυδονητή Β, η $v_3=0$ ισοδυναμεί με $v_2$ στη χαμηλή στάθμη. Επομένως, η διάρκεια της κλίμακας είναι \begin{equation*}
		t_{\mathrm{L}}=R_B\cdot C_2\cdot\ln{\left[\frac{1-\beta\(\sfrac{L_+}{L_-}\)}{1-\beta}\right]}
	\end{equation*}
	και η απόσταση μεταξύ των κλιμάκων είναι \begin{equation*}
		t_{\mathrm{H}}=R_A\cdot C_2\cdot\ln{\left[\frac{1-\beta\(\sfrac{L_-}{L_+}\)}{1-\beta}\right]}.
	\end{equation*}

	Το ύψος κάθε βήματος καθορίζεται από τον ολοκληρωτή ως εξής
	\begin{equation*}
		V_{\mathrm{step}}=\frac{2}{R\cdot C}\int_{t_a}^{t_b}{v_1(t)\dd{t}},
	\end{equation*}
	όπου $t_a$ είναι μια χρονική στιγμή στην οποία η $v_1$ περνάει από μηδέν σε υψηλή σταθμη και $t_b=t_a+t_{\mathrm{on}}$.\par